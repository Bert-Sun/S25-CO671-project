% 🇩🇪 wird fuer Tabellen benötigt (z.B. >{centering\RBS}p{2.5cm} erzeugt einen zentrierten 2,5cm breiten Absatz in einer Tabelle
\newcommand{\RBS}{\let\\=\tabularnewline}

% 🇺🇸 To avoid issues with Springer's \mathplus. See also http://tex.stackexchange.com/q/212644/9075
\providecommand\mathplus{+}

% 🇺🇸 from hmks makros.tex - \indexify
\newcommand{\toindex}[1]{\index{#1}#1}

% 🇩🇪 Tipp aus "The Comprehensive LaTeX Symbol List"
\newcommand{\dotcup}{\ensuremath{\,\mathaccent\cdot\cup\,}}

% 🇩🇪 Anstatt $|x|$ $\abs{x}$ verwenden. Die Betragsstriche skalieren automatisch, falls "x" etwas größer sein sollte...
% \newcommand{\abs}[1]{\left\lvert#1\right\rvert}

% 🇩🇪 Seitengrößen - Gegen Schusterjungen und Hurenkinder...
\newcommand{\largepage}{\enlargethispage{\baselineskip}}
\newcommand{\shortpage}{\enlargethispage{-\baselineskip}}

\newcommand{\initialism}[1]{%
  \textlcc{#1}\xspace%
}
\newcommand{\OMG}{\initialism{OMG}}
\newcommand{\BPEL}{\initialism{BPEL}}
\newcommand{\BPMN}{\initialism{BPMN}}
\newcommand{\UML}{\initialism{UML}}

\newcommand{\En}{\ensuremath{\mathbb{N}}}
\newcommand{\Que}{\ensuremath{\mathbb{Q}}}
\newcommand{\Ree}{\ensuremath{\mathbb{R}}}
\newcommand{\Zee}{\ensuremath{\mathbb{Z}}}
\newcommand{\Kay}{\ensuremath{\mathbb{K}}}
\newcommand{\Cee}{\ensuremath{\mathbb{C}}}
\newcommand{\Tee}{\ensuremath{\mathbb{T}}}
\newcommand{\Eff}{\ensuremath{\mathbb{F}}}
\newcommand{\Ess}{\ensuremath{\mathbb{S}}}
\newcommand{\Essp}{\ensuremath{\mathbb{S}_{+}}}
\newcommand{\Esspp}{\ensuremath{\mathbb{S}_{++}}}
\newcommand{\Reep}{\ensuremath{\mathbb{R}_+}}
\newcommand{\Reepp}{\ensuremath{\mathbb{R}_{++}}}
\newcommand{\Ach}{\ensuremath{\mathbb{H}}}

\newcommand{\mbP}{\ensuremath{\mathbb{P}}}
\newcommand{\mbA}{\ensuremath{\mathbb{A}}}

\newcommand{\mcF}{\ensuremath{\mathcal{F}}}
\newcommand{\mcP}{\ensuremath{\mathcal{P}}}
\newcommand{\mcB}{\ensuremath{\mathcal{B}}}
\newcommand{\mcA}{\ensuremath{\mathcal{A}}}
\newcommand{\mcS}{\ensuremath{\mathcal{S}}}
\newcommand{\mcH}{\ensuremath{\mathcal{H}}}
\newcommand{\mcK}{\ensuremath{\mathcal{K}}}
\newcommand{\mcC}{\ensuremath{\mathcal{C}}}
\newcommand{\mcL}{\ensuremath{\mathcal{L}}}
\newcommand{\mcG}{\ensuremath{\mathcal{G}}}
\newcommand{\mcN}{\ensuremath{\mathcal{N}}}
\newcommand{\mcQ}{\ensuremath{\mathcal{Q}}}

\newcommand*{\eps}{\ensuremath{\epsilon}}

\newcommand*{\one}{\text{\usefont{U}{bbold}{m}{n}1}}
\newcommand{\zero}{\text{\usefont{U}{bbold}{m}{n}0}}

\newcommand{\tnorm}[1]{|\!|\!|#1|\!|\!|}

\newcommand{\wbar}[1]{\overline{#1}}

\newcommand{\imps}{\Rightarrow}
\newcommand{\Iff}{\Leftrightarrow}
\newcommand{\limnfty}[1][n]{\lim_{#1\to\infty}}
% \NewDocumentCommand{\limnfty}{O{n}}{\lim_{#1\to\infty}}
\newcommand{\suminf}[1]{\sum_{#1=1}^{\infty}}
\newcommand{\sumto}[2]{\sum_{#1=1}^{#2}}
\newcommand{\seqnfty}[1][x]{(#1_n)_{n=1}^\infty}

\newcommand{\taninv}{\tan^{-1}}

\DeclareMathOperator*{\argmax}{arg\,max}
\DeclareMathOperator*{\argmin}{arg\,min}

\DeclareMathOperator*{\diam}{diam}
\newcommand{\inv}[1]{#1^{-1}}

\DeclareMathOperator{\dist}{dist}
\DeclareMathOperator{\diag}{diag}
\DeclareMathOperator{\Diag}{Diag}
\DeclareMathOperator{\spr}{spr}
\DeclareMathOperator{\Ind}{Ind}
\DeclareMathOperator{\ctrl}{c-}
\DeclareMathOperator{\lcm}{lcm}
\DeclareMathOperator{\spann}{span}
\DeclareMathOperator{\ran}{ran}
\DeclareMathOperator{\tr}{tr}
\DeclareMathOperator{\Tr}{Tr}
\DeclareMathOperator{\rank}{rank}
\DeclareMathOperator{\nullity}{nullity}
\DeclareMathOperator{\nullsp}{Null}
\DeclareMathOperator{\ext}{ext}
\DeclareMathOperator{\Ext}{Ext}
\DeclareMathOperator{\re}{Re}
\DeclareMathOperator{\im}{Im}
\DeclareMathOperator{\GL}{GL}
\DeclareMathOperator{\relint}{relint}
\DeclareMathOperator{\Aff}{Aff}

\DeclarePairedDelimiter\abs{\lvert}{\rvert}
\DeclarePairedDelimiter\norm{\lVert}{\rVert}
\DeclarePairedDelimiter\bra{\langle}{\rvert}
\DeclarePairedDelimiter\ket{\lvert}{\rangle}
\DeclarePairedDelimiter\ceil{\lceil}{\rceil}
\DeclarePairedDelimiterX\ip[2]{\langle}{\rangle}{#1\,\delimsize\vert\,\mathopen{}#2}
\DeclarePairedDelimiterX\op[2]{\lvert}{\rvert}{\mathopen{}#1\delimsize\rangle\delimsize\langle\mathopen{}#2}
\newcommand{\opproj}[1]{\op{#1}{#1}}
\DeclareMathOperator*{\QFT}{QFT}
\DeclareMathOperator*{\Adv}{Adv}


\newcommand{\ko}{\ket{0}}
\newcommand{\ki}{\ket{1}}
\newcommand{\koo}{\ket{00}}
\newcommand{\koi}{\ket{01}}
\newcommand{\kio}{\ket{10}}
\newcommand{\kii}{\ket{11}}
\newcommand{\cnot}{\textsc{cnot}}
