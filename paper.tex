% This template has been tested with LLNCS DOCUMENT CLASS -- version 2.21 (12-Jan-2022)

% !TeX spellcheck = en-US
% LTeX: language=en-US
% !TeX encoding = utf8
% !TeX program = lualatex
% !BIB program = bibtex
% -*- coding:utf-8 mod:LaTeX -*-

% "a4paper" enables:
%
%  - easy print out on DIN A4 paper size
%
% One can configure default page size (a4 vs. letter) in the LaTeX installation.
% Thus, it is configuration dependend, what the paper size will be.
% Having "a4paper" option present, the page size is set to A4.
% Note that the current word template offered by Springer is DIN A4.
%
% "runningheads" enables:
%
%  - page number on page 2 onwards
%  - title/authors on even/odd pages
%
% This is good for other readers to enable proper archiving among other papers and pointing to
% content. Even if the title page states the title, when printed and stored in a folder, when
% blindly opening the folder, one could hit not the title page, but an arbitrary page. Therefore,
% it is good to have title printed on the pages, too.
%
% The optiion "runningheads" neesd to be removed upon request of the publisher.
%
% To disable outputting page headers and footers, remove "runningheads"
\documentclass[runningheads,a4paper,english]{llncs}[2022/01/12]

\usepackage{amsmath, amssymb, mathtools}

\usepackage{iftex}

% backticks (`) are rendered as such in verbatim environments.
% See following links for details:
%   - https://tex.stackexchange.com/a/341057/9075
%   - https://tex.stackexchange.com/a/47451/9075
%   - https://tex.stackexchange.com/a/166791/9075
\usepackage{upquote}

% Set English as language and allow to write hyphenated"=words
%
% Even though `american`, `english` and `USenglish` are synonyms for babel package (according to https://tex.stackexchange.com/questions/12775/babel-english-american-usenglish), the llncs document class is prepared to avoid the overriding of certain names (such as "Abstract." -> "Abstract" or "Fig." -> "Figure") when using `english`, but not when using the other 2.
% english has to go last to set it as default language
\usepackage[ngerman,main=english]{babel}
%
% Hint by http://tex.stackexchange.com/a/321066/9075 -> enable "= as dashes
\addto\extrasenglish{\languageshorthands{ngerman}\useshorthands{"}}

% Links behave as they should. Enables "\url{...}" for URL typesettings.
% Allow URL breaks also at a hyphen, even though it might be confusing: Is the "-" part of the address or just a hyphen?
% See https://tex.stackexchange.com/a/3034/9075.
\usepackage[hyphens]{url}

% When activated, use text font as url font, not the monospaced one.
% For all options see https://tex.stackexchange.com/a/261435/9075.
% \urlstyle{same}

% Improve wrapping of URLs - hint by http://tex.stackexchange.com/a/10419/9075
\makeatletter
\g@addto@macro{\UrlBreaks}{\UrlOrds}
\makeatother

% nicer // - solution by http://tex.stackexchange.com/a/98470/9075
% DO NOT ACTIVATE -> prevents line breaks
%\makeatletter
%\def\Url@twoslashes{\mathchar`\/\@ifnextchar/{\kern-.2em}{}}
%\g@addto@macro\UrlSpecials{\do\/{\Url@twoslashes}}
%\makeatother

%% !!! If you change the font, be sure that words such as "workflow" can
%% !!! still be copied from the PDF. If this is not the case, you have
%% !!! to use glyphtounicode. See comment at cmap package.
%%
%% Background: "workflow" contains "fl" which is a ligature, which in turn
%%             is rendered as one character in the PDF and needs to be split
%%             whily copying.

\ifluatex
  \usepackage[no-math]{fontspec}
  \usepackage{unicode-math}

  % Typewriter font (for source code etc)
  % Use New Computer Modern font (Computer Modern is the default LaTeX font; this is the implemented modern variant)
  % Source: https://tug.org/FontCatalogue/newcomputermoderntypewriter/

  \setmainfont[
    ItalicFont=NewCM10-Italic.otf,
    BoldFont=NewCM10-Bold.otf,
    BoldItalicFont=NewCM10-BoldItalic.otf,
    SmallCapsFeatures={Numbers=OldStyle}]{NewCM10-Regular.otf}

  \setsansfont[
    ItalicFont=NewCMSans10-Oblique.otf,
    BoldFont=NewCMSans10-Bold.otf,
    BoldItalicFont=NewCMSans10-BoldOblique.otf,
    SmallCapsFeatures={Numbers=OldStyle}]{NewCMSans10-Regular.otf}

  \setmonofont[ItalicFont=NewCMMono10-Italic.otf,
    BoldFont=NewCMMono10-Bold.otf,
    BoldItalicFont=NewCMMono10-BoldOblique.otf,
    SmallCapsFeatures={Numbers=OldStyle}]{NewCMMono10-Regular.otf}

  \setmathfont{NewCMMath-Regular.otf}

  % Enable proper ligatures
  % For more information see https://ctan.org/pkg/selnolig
  % language "english" or "ngerman" is passed to selnolig by the document class
  \usepackage{selnolig}

\else
  % This is the modern package for "Computer Modern".
  % In case this gets activated, one has to switch from cmap package to glyphtounicode (in the case of pdflatex)
  \usepackage[%
    rm={oldstyle=false,proportional=true},%
    sf={oldstyle=false,proportional=true},%
    % By using 'variable=true' the monospaced font can be used as variable font (with differents widths per letter)
    % However, this makes listings look ugly.
    tt={oldstyle=false,proportional=true,variable=false},%
    qt=false%
  ]{cfr-lm}

  % Has to be loaded AFTER any font packages. See https://tex.stackexchange.com/a/2869/9075.
  \usepackage[T1]{fontenc}
\fi

% Character protrusion and font expansion. See http://www.ctan.org/tex-archive/macros/latex/contrib/microtype/

\usepackage[
  babel=true, % Enable language-specific kerning. Take language-settings from the languge of the current document (see Section 6 of microtype.pdf)
  expansion=alltext,
  protrusion=alltext-nott, % Ensure that at listings, there is no change at the margin of the listing
  % In the standard configuration, this template is always in the final mode, so this option only makes a difference if "pros" use the draft mode
  final % Always enable microtype, even if in draft mode. This helps finding bad boxes quickly.
]{microtype}

% \texttt{test -- test} keeps the "--" as "--" (and does not convert it to an en dash)
\DisableLigatures{encoding = T1, family = tt* }

%\DeclareMicrotypeSet*[tracking]{my}{ font = */*/*/sc/* }%
%\SetTracking{ encoding = *, shape = sc }{ 45 }
% Source: http://homepage.ruhr-uni-bochum.de/Georg.Verweyen/pakete.html
% Deactiviated, because does not look good

\usepackage{graphicx}

% Diagonal lines in a table - http://tex.stackexchange.com/questions/17745/diagonal-lines-in-table-cell
% Slashbox is not available in texlive (due to licensing) and also gives bad results. Thus, we use diagbox
\usepackage{diagbox}

\ifluatex
  \usepackage{spelling}
  \spellingoutput{off}
\fi

\usepackage[dvipsnames, table]{xcolor}
% Code Listings
\usepackage{listings}

\definecolor{eclipseStrings}{RGB}{42,0.0,255}
\definecolor{eclipseKeywords}{RGB}{127,0,85}
\colorlet{numb}{magenta!60!black}

% JSON definition
% Source: https://tex.stackexchange.com/a/433961/9075

\lstdefinelanguage{json}{
  basicstyle=\normalfont\ttfamily,
  commentstyle=\color{eclipseStrings}, % style of comment
  stringstyle=\color{eclipseKeywords}, % style of strings
  numbers=left,
  numberstyle=\scriptsize,
  stepnumber=1,
  numbersep=8pt,
  showstringspaces=false,
  breaklines=true,
  frame=lines,
  % backgroundcolor=\color{gray}, %only if you like
  string=[s]{"}{"},
  comment=[l]{:\ "},
  morecomment=[l]{:"},
  literate=
    *{0}{{{\color{numb}0}}}{1}
    {1}{{{\color{numb}1}}}{1}
    {2}{{{\color{numb}2}}}{1}
    {3}{{{\color{numb}3}}}{1}
    {4}{{{\color{numb}4}}}{1}
    {5}{{{\color{numb}5}}}{1}
    {6}{{{\color{numb}6}}}{1}
    {7}{{{\color{numb}7}}}{1}
    {8}{{{\color{numb}8}}}{1}
    {9}{{{\color{numb}9}}}{1}
}

\lstset{
  % everything between (* *) is a latex command
  escapeinside={(*}{*)},
  %
  language=json,
  %
  showstringspaces=false,
  %
  extendedchars=true,
  %
  basicstyle=\footnotesize\ttfamily,
  %
  commentstyle=\slshape,
  %
  % default: \rmfamily
  stringstyle=\ttfamily,
  %
  breaklines=true,            % Zeilen werden umbrochen
  %
  breakatwhitespace=true,
  %
  % alternative: fixed
  columns=flexible,
  %
  tabsize=2,                  % Groesse von Tabs
  %
  numbers=left,
  %
  numberstyle=\tiny,
  %
  basewidth=.5em,
  %
  xleftmargin=.5cm,
  %
  % aboveskip=0mm,
  %
  % belowskip=0mm,
  %
  captionpos=b
}
\ifpdftex

  % Enable Umlauts when using \lstinputputlisting.
  % See https://stackoverflow.com/a/29260603/873282 für details.
  % listingsutf8 did not work in June 2020.
  \lstset{literate=
    {á}{{\'a}}1 {é}{{\'e}}1 {í}{{\'i}}1 {ó}{{\'o}}1 {ú}{{\'u}}1
  {Á}{{\'A}}1 {É}{{\'E}}1 {Í}{{\'I}}1 {Ó}{{\'O}}1 {Ú}{{\'U}}1
  {à}{{\`a}}1 {è}{{\`e}}1 {ì}{{\`i}}1 {ò}{{\`o}}1 {ù}{{\`u}}1
  {À}{{\`A}}1 {È}{{\'E}}1 {Ì}{{\`I}}1 {Ò}{{\`O}}1 {Ù}{{\`U}}1
  {ä}{{\"a}}1 {ë}{{\"e}}1 {ï}{{\"i}}1 {ö}{{\"o}}1 {ü}{{\"u}}1
  {Ä}{{\"A}}1 {Ë}{{\"E}}1 {Ï}{{\"I}}1 {Ö}{{\"O}}1 {Ü}{{\"U}}1
  {â}{{\^a}}1 {ê}{{\^e}}1 {î}{{\^i}}1 {ô}{{\^o}}1 {û}{{\^u}}1
  {Â}{{\^A}}1 {Ê}{{\^E}}1 {Î}{{\^I}}1 {Ô}{{\^O}}1 {Û}{{\^U}}1
  {Ã}{{\~A}}1 {ã}{{\~a}}1 {Õ}{{\~O}}1 {õ}{{\~o}}1
  {œ}{{\oe}}1 {Œ}{{\OE}}1 {æ}{{\ae}}1 {Æ}{{\AE}}1 {ß}{{\ss}}1
  {ű}{{\H{u}}}1 {Ű}{{\H{U}}}1 {ő}{{\H{o}}}1 {Ő}{{\H{O}}}1
  {ç}{{\c c}}1 {Ç}{{\c C}}1 {ø}{{\o}}1 {å}{{\r a}}1 {Å}{{\r A}}1
  }
\fi

\lstloadlanguages{% Check dokumentation for further languages...
  %[Visual]Basic
  %Pascal
  %C
  %C++
  %XML
  %HTML
}

% For easy quotations: \enquote{text}
% This package is very smart when nesting is applied, otherwise textcmds (see below) provides a shorter command
\usepackage[autostyle=true]{csquotes}

% Enable using "`quote"' - see https://tex.stackexchange.com/a/150954/9075
\defineshorthand{"`}{\openautoquote}
\defineshorthand{"'}{\closeautoquote}

% Nicer tables (\toprule, \midrule, \bottomrule)
\usepackage{booktabs}

% Extended enumerate, such as \begin{compactenum}
\usepackage{paralist}

% Bibliopgraphy enhancements
%  - enable \cite[prenote][]{ref}
%  - enable \cite{ref1,ref2}
% Alternative: \usepackage{cite}, which enables \cite{ref1, ref2} only (otherwise: Error message: "White space in argument")

% Doc: http://texdoc.net/natbib
\usepackage[%
  square,        % for square brackets
  comma,         % use commas as separators
  numbers,       % for numerical citations;
  %sort           % orders multiple citations into the sequence in which they appear in the list of references;
  sort&compress  % as sort but in addition multiple numerical citations are compressed if possible (as 3-6, 15);
]{natbib}

% In the bibliography, references have to be formatted as 1., 2., ... not [1], [2], ...
\renewcommand{\bibnumfmt}[1]{#1.}

% Enable hyperlinked author names in the case of \citet
% Source: https://tex.stackexchange.com/a/76075/9075
\usepackage{etoolbox}
\makeatletter
\patchcmd{\NAT@test}{\else \NAT@nm}{\else \NAT@hyper@{\NAT@nm}}{}{}
\makeatother

% Prepare more space-saving rendering of the bibliography
% Source: https://tex.stackexchange.com/a/280936/9075
\SetExpansion
[ context = sloppy,
  stretch = 30,
  shrink = 60,
  step = 5 ]
{ encoding = {OT1,T1,TS1} }
{ }

% Put figures aside a text
% Even though the package is from 1998, it works well
\usepackage[rflt]{floatflt}

% Farbige Tabellen
% ----------------
% Das Paket colortbl wird inzwischen automatisch durch xcolor geladen
%
% Erweiterte Funktionen innerhalb von Tabellen
% --------------------------------------------
%%% Doc: http://mirror.ctan.org/tex-archive/macros/latex/contrib/multirow/multirow.sty
\usepackage{multirow} % Mehrfachspalten
%
%%% Doc: Documentation inside dtx Package
\usepackage{dcolumn}  % Ausrichtung an Komma oder Punkt

%%% Doc: http://mirror.ctan.org/tex-archive/macros/latex/contrib/supertabular/supertabular.pdf
%\usepackage{supertabular}

%%% Fussnoten/Endnoten ===================================================

% EN: Put footnotes below floats
% DE: Fußnoten unter Gleitumgebungen ("floats") platzieren
% Source: https://tex.stackexchange.com/a/32993/9075
\usepackage{stfloats}
\fnbelowfloat

% EN: Extended support for footnotes
% DE: Fußnoten
%
%\usepackage{dblfnote}  %Zweispaltige Fußnoten
%
% Keine hochgestellten Ziffern in der Fußnote (KOMA-Script-spezifisch):
%\deffootnote[1.5em]{0pt}{1em}{\makebox[1.5em][l]{\bfseries\thefootnotemark}}
%
% Abstand zwischen Fußnoten vergrößern:
%\setlength{\footnotesep}{.85\baselineskip}
%
% EN: Following command disables the separting line of the footnote
% DE: Folgendes Kommando deaktiviert die Trennlinie zur Fußnote
%\renewcommand{\footnoterule}{}
%
%\addtolength{\skip\footins}{\baselineskip} % Abstand Text <-> Fußnote

% DE: Fußnoten immer ganz unten auf einer \raggedbottom-Seite
% DE: fnpos kommt aus dem yafoot package
%\usepackage{fnpos}
%\makeFNbelow
%\makeFNbottom

% TODO (and comment) configuration
%
% - \todo (from todo, easy-todo, todonotes) / \TODO (from fixmetodonotes) - for "normal" TODOs
% - \todofix - "important" TODOs
%
% - \textcomment - highlights text and has a hover comment
% - \sidecomment - just puts a comment to the side. Note: \comment MUST NOT be used as command name, it is already defined by much packages (mathdesign, mindflow, verbatim, and others)
%
% - \missingfigure
%
% - \textmarker
% - \modified
% - \change      - adresses a review comment

% Enable nice comments
\usepackage{pdfcomment}

\newcommand{\textcomment}[2]{\colorbox{yellow!60}{#1}\pdfcomment[color={0.234 0.867 0.211},hoffset=-6pt,voffset=10pt,opacity=0.5]{#2}}

% Small PDF comment
% 1. Parameter: Comment
\newcommand{\sidecomment}[1]{\pdfcomment[color={0.045 0.278 0.643},voffset=4pt,icon=Note]{#1}}
% Disabled variant - for the final PDF
%\newcommand{\sidecomment}[1]{}

\newcommand{\todo}[1]{TODO!\sidecomment{#1}}

% Änderungen
%
% 1. Parameter: Review-Kommentar
% 2. Parameter: Neuer Text
\newcommand{\change}[2]{{\color{red}#2}\pdfcomment[color={0.234 0.867 0.211},voffset=8pt,opacity=0.5]{#1}}
% Disabled variant - for the final PDF
%\newcommand{\change}[2]{#2}

% Define default commands
\makeatletter
\@ifundefined{missingfigure}{\newcommand{\missingfigure}{... missing figure ...}}{}
\@ifundefined{textcomment}{\newcommand{\textcomment}[2]{#1 \todo{#2}}}{}
\@ifundefined{sidecomment}{\newcommand{\sidecomment}[1]{\marginpar{#1}}}{}
\@ifundefined{todo}{\newcommand{\todo}[1]{\sidecomment{#1}}}{}
\@ifundefined{TODO}{\newcommand{\TODO}[1]{\todo{#1}}}{}
\@ifundefined{todofix}{\newcommand{\todofix}[1]{\todo{#1}}}{}
\@ifundefined{change}{\newcommand{\change}[2]{#1 $\rightarrow$ #2}}{}
\makeatother

% Textmarker (Textfarbe rot)
\newcommand{\textmarker}[1]{{\color{red} #1}\xspace}

% Modified (Text blau)
\newcommand{\modified}[1]{{\color{blue!60!black} #1}\xspace}

\usepackage[group-minimum-digits=4,per-mode=fraction]{siunitx}

% Enable that parameters of \cref{}, \ref{}, \cite{}, ... are linked so that a reader can click on the number an jump to the target in the document
\usepackage{hyperref}

% Enable hyperref without colors and without bookmarks
\hypersetup{
  hidelinks,
  colorlinks=true,       % Links erhalten Farben statt Kaeten
  raiselinks=true,       % calculate real height of the link
  allcolors=black,
  pdfstartview=Fit,
  breaklinks=true,       % Links ueberstehen Zeilenumbruch
  hypertexnames=false,   % Fix jumping to algorithm line - http://tex.stackexchange.com/a/156404/9075
}

% Enable correct jumping to figures when referencing
\usepackage[all]{hypcap}

\usepackage[caption=false,font=footnotesize]{subfig}

% Alternative for making subfigures:
% Part of the caption package. See http://www.ctan.org/pkg/caption
% Ersetzt die Pakete subfigure und subfig - siehe https://tex.stackexchange.com/a/13778/9075
%
% (subfigure is outdated. subfig is maintained, but subcaption is better)
% See: http://tex.stackexchange.com/questions/13625/subcaption-vs-subfig-best-package-for-referencing-a-subfigure
%\usepackage[hypcap=true]{subcaption}

\usepackage{mindflow}

% Extensions for references inside the document (\cref{fig:sample}, ...)
% Enable usage \cref{...} and \Cref{...} instead of \ref: Type of reference included in the link
% That means, "Figure 5" is a full link instead of just "5".
\usepackage[capitalise,nameinlink]{cleveref}

\crefname{section}{Sect.}{Sect.}
\Crefname{section}{Section}{Sections}
\crefname{listing}{List.}{List.}
\crefname{listing}{Listing}{Listings}
\Crefname{listing}{Listing}{Listings}
\crefname{lstlisting}{Listing}{Listings}
\Crefname{lstlisting}{Listing}{Listings}

\usepackage{lipsum}

% For demonstration purposes only
% These packages can be removed when all examples have been deleted
\usepackage[math]{blindtext}
\usepackage{mwe}
\usepackage[realmainfile]{currfile}
\usepackage{tcolorbox}
\tcbuselibrary{listings}

%introduce \powerset - hint by http://matheplanet.com/matheplanet/nuke/html/viewtopic.php?topic=136492&post_id=997377
\DeclareFontFamily{U}{MnSymbolC}{}
\DeclareSymbolFont{MnSyC}{U}{MnSymbolC}{m}{n}
\DeclareFontShape{U}{MnSymbolC}{m}{n}{
  <-6>    MnSymbolC5
  <6-7>   MnSymbolC6
  <7-8>   MnSymbolC7
  <8-9>   MnSymbolC8
  <9-10>  MnSymbolC9
  <10-12> MnSymbolC10
  <12->   MnSymbolC12%
}{}
\DeclareMathSymbol{\powerset}{\mathord}{MnSyC}{180}

% Allows for defining commands that don't eat spaces.
\usepackage{xspace}
% Adds compatibility to \xspace und \enquote
\makeatletter
\xspaceaddexceptions{\grqq \grq \csq@qclose@i \} }
\makeatother

\newcommand{\eg}{e.g.,\ }
\newcommand{\ie}{i.e.,\ }

% Enable hyphenation at other places as the dash.
% Example: applicaiton\hydash specific
\makeatletter
\newcommand{\hydash}{\penalty\@M-\hskip\z@skip}
% Definition of "= taken from http://mirror.ctan.org/macros/latex/contrib/babel-contrib/german/ngermanb.dtx
\makeatother

% Add manual adapted hyphenation of English words
% See https://ctan.org/pkg/hyphenex and https://tex.stackexchange.com/a/22892/9075 for details
\input{ushyphex}

% correct bad hyphenation here
\hyphenation{
  op-tical net-works semi-conduc-tor
  % May not be hypphenated
  AROMA TOSCA BPMN OASIS OMG DMTF IT DevOps
}
% 🇩🇪 wird fuer Tabellen benötigt (z.B. >{centering\RBS}p{2.5cm} erzeugt einen zentrierten 2,5cm breiten Absatz in einer Tabelle
\newcommand{\RBS}{\let\\=\tabularnewline}

% 🇺🇸 To avoid issues with Springer's \mathplus. See also http://tex.stackexchange.com/q/212644/9075
\providecommand\mathplus{+}

% 🇺🇸 from hmks makros.tex - \indexify
\newcommand{\toindex}[1]{\index{#1}#1}

% 🇩🇪 Tipp aus "The Comprehensive LaTeX Symbol List"
\newcommand{\dotcup}{\ensuremath{\,\mathaccent\cdot\cup\,}}

% 🇩🇪 Anstatt $|x|$ $\abs{x}$ verwenden. Die Betragsstriche skalieren automatisch, falls "x" etwas größer sein sollte...
% \newcommand{\abs}[1]{\left\lvert#1\right\rvert}

% 🇩🇪 Seitengrößen - Gegen Schusterjungen und Hurenkinder...
\newcommand{\largepage}{\enlargethispage{\baselineskip}}
\newcommand{\shortpage}{\enlargethispage{-\baselineskip}}

\newcommand{\initialism}[1]{%
  \textlcc{#1}\xspace%
}
\newcommand{\OMG}{\initialism{OMG}}
\newcommand{\BPEL}{\initialism{BPEL}}
\newcommand{\BPMN}{\initialism{BPMN}}
\newcommand{\UML}{\initialism{UML}}

\newcommand{\En}{\ensuremath{\mathbb{N}}}
\newcommand{\Que}{\ensuremath{\mathbb{Q}}}
\newcommand{\Ree}{\ensuremath{\mathbb{R}}}
\newcommand{\Zee}{\ensuremath{\mathbb{Z}}}
\newcommand{\Kay}{\ensuremath{\mathbb{K}}}
\newcommand{\Cee}{\ensuremath{\mathbb{C}}}
\newcommand{\Tee}{\ensuremath{\mathbb{T}}}
\newcommand{\Eff}{\ensuremath{\mathbb{F}}}
\newcommand{\Ess}{\ensuremath{\mathbb{S}}}
\newcommand{\Essp}{\ensuremath{\mathbb{S}_{+}}}
\newcommand{\Esspp}{\ensuremath{\mathbb{S}_{++}}}
\newcommand{\Reep}{\ensuremath{\mathbb{R}_+}}
\newcommand{\Reepp}{\ensuremath{\mathbb{R}_{++}}}
\newcommand{\Ach}{\ensuremath{\mathbb{H}}}

\newcommand{\mbP}{\ensuremath{\mathbb{P}}}
\newcommand{\mbA}{\ensuremath{\mathbb{A}}}
\newcommand{\mbE}{\ensuremath{\mathbb{E}}}

\newcommand{\mcF}{\ensuremath{\mathcal{F}}}
\newcommand{\mcP}{\ensuremath{\mathcal{P}}}
\newcommand{\mcB}{\ensuremath{\mathcal{B}}}
\newcommand{\mcA}{\ensuremath{\mathcal{A}}}
\newcommand{\mcS}{\ensuremath{\mathcal{S}}}
\newcommand{\mcH}{\ensuremath{\mathcal{H}}}
\newcommand{\mcK}{\ensuremath{\mathcal{K}}}
\newcommand{\mcC}{\ensuremath{\mathcal{C}}}
\newcommand{\mcL}{\ensuremath{\mathcal{L}}}
\newcommand{\mcG}{\ensuremath{\mathcal{G}}}
\newcommand{\mcN}{\ensuremath{\mathcal{N}}}
\newcommand{\mcQ}{\ensuremath{\mathcal{Q}}}
\newcommand{\mcE}{\ensuremath{\mathcal{E}}}

\newcommand*{\eps}{\ensuremath{\epsilon}}

\newcommand*{\one}{\text{\usefont{U}{bbold}{m}{n}1}}
\newcommand{\zero}{\text{\usefont{U}{bbold}{m}{n}0}}

\newcommand{\tnorm}[1]{|\!|\!|#1|\!|\!|}

\newcommand{\wbar}[1]{\overline{#1}}

\newcommand{\imps}{\Rightarrow}
\newcommand{\Iff}{\Leftrightarrow}
\newcommand{\limnfty}[1][n]{\lim_{#1\to\infty}}
% \NewDocumentCommand{\limnfty}{O{n}}{\lim_{#1\to\infty}}
\newcommand{\suminf}[1]{\sum_{#1=1}^{\infty}}
\newcommand{\sumto}[2]{\sum_{#1=1}^{#2}}
\newcommand{\seqnfty}[1][x]{(#1_n)_{n=1}^\infty}

\newcommand{\taninv}{\tan^{-1}}

\DeclareMathOperator*{\argmax}{arg\,max}
\DeclareMathOperator*{\argmin}{arg\,min}

\DeclareMathOperator*{\diam}{diam}
\newcommand{\inv}[1]{#1^{-1}}

\DeclareMathOperator{\dist}{dist}
\DeclareMathOperator{\diag}{diag}
\DeclareMathOperator{\Diag}{Diag}
\DeclareMathOperator{\spr}{spr}
\DeclareMathOperator{\Ind}{Ind}
\DeclareMathOperator{\ctrl}{c-}
\DeclareMathOperator{\lcm}{lcm}
\DeclareMathOperator{\spann}{span}
\DeclareMathOperator{\ran}{ran}
\DeclareMathOperator{\tr}{tr}
\DeclareMathOperator{\Tr}{Tr}
\DeclareMathOperator{\rank}{rank}
\DeclareMathOperator{\nullity}{nullity}
\DeclareMathOperator{\nullsp}{Null}
\DeclareMathOperator{\ext}{ext}
\DeclareMathOperator{\Ext}{Ext}
\DeclareMathOperator{\re}{Re}
\DeclareMathOperator{\im}{Im}
\DeclareMathOperator{\GL}{GL}
\DeclareMathOperator{\relint}{relint}
\DeclareMathOperator{\conv}{conv}
\DeclareMathOperator{\Aff}{Aff}
\DeclareMathOperator{\Dens}{Dens}
\DeclareMathOperator{\Lin}{Lin}
\DeclareMathOperator{\Sep}{Sep}
% \DeclareMathOperator{\op}{op}
\DeclareMathOperator{\poly}{poly}
\DeclareMathOperator{\Image}{Image}
\DeclareMathOperator{\OPT}{OPT}

\DeclarePairedDelimiter\abs{\lvert}{\rvert}
\DeclarePairedDelimiter\norm{\lVert}{\rVert}
\DeclarePairedDelimiter\bra{\langle}{\rvert}
\DeclarePairedDelimiter\ket{\lvert}{\rangle}
\DeclarePairedDelimiter\ceil{\lceil}{\rceil}
\DeclarePairedDelimiterX\ip[2]{\langle}{\rangle}{#1\,\delimsize\vert\,\mathopen{}#2}
\DeclarePairedDelimiterX\op[2]{\lvert}{\rvert}{\mathopen{}#1\delimsize\rangle\delimsize\langle\mathopen{}#2}
\newcommand{\opproj}[1]{\op{#1}{#1}}
\DeclareMathOperator*{\QFT}{QFT}
\DeclareMathOperator*{\Adv}{Adv}


\newcommand{\ko}{\ket{0}}
\newcommand{\ki}{\ket{1}}
\newcommand{\koo}{\ket{00}}
\newcommand{\koi}{\ket{01}}
\newcommand{\kio}{\ket{10}}
\newcommand{\kii}{\ket{11}}
\newcommand{\cnot}{\textsc{cnot}}


% Add copyright
%
% This is recommended if you intend to send the version to colleagues
% See https://ctan.org/pkg/llncsconf for details
\iffalse
  % state: intended | submitted | llncs
  % you can add "crop" if the paper should be cropped to the format Springer is publishing
  \usepackage[intended]{llncsconf}

  \conference{name of the conference}

  % in case of "proceedings" (final version!)
  % example: \llncs{Anonymous et al. (eds). \emph{Proceedings of the International Conference on \LaTeX-Hacks}, LNCS~42. Some Publisher, 2016.}{0042}
  % 0042 denotes an example start page
  \llncs{book editors and title}{0042}
\fi

\ifpdftex
  % Enable copy and paste of text from the PDF
  % Only required for pdflatex. It "just works" in the case of lualatex.
  % Alternative: cmap or mmap package
  % mmap enables mathematical symbols, but does not work with the newtx font set
  % See: https://tex.stackexchange.com/a/64457/9075
  % Other solutions outlined at http://goemonx.blogspot.de/2012/01/pdflatex-ligaturen-und-copynpaste.html and http://tex.stackexchange.com/questions/4397/make-ligatures-in-linux-libertine-copyable-and-searchable
  % Trouble shooting outlined at https://tex.stackexchange.com/a/100618/9075
  %
  % According to https://tex.stackexchange.com/q/451235/9075 this is the way to go
  \input{glyphtounicode}
  \pdfgentounicode=1
\fi
\begin{document}

\title{CO 671 Final Project on Semidefinite Programming Integrality Gaps in Quantum Information}
% If Title is too long, use \titlerunning
\titlerunning{SDP Integrality Gaps in Quantum Information}

% Single insitute
\author{Bert Sun}

% If there are too many authors, use \authorrunning
%\authorrunning{First Author et al.}

\institute{University of Waterloo}

%% Multiple insitutes - ALTERNATIVE to the above
% \author{%
%     Firstname Lastname\inst{1} \and
%     Firstname Lastname\inst{2}
% }
%
%If there are too many authors, use \authorrunning
%  \authorrunning{First Author et al.}
%
%  \institute{
%      Insitute 1\\
%      \email{...}\and
%      Insitute 2\\
%      \email{...}
%}

\maketitle

% \begin{abstract}
% \lipsum[1]

% % \keywords{First keyword \and Second keyword \and Third keyword}
% \end{abstract}


\section{Introduction}
\label{sec:introduction}
Semidefinite Programs (SDP) provide a modelling framework to optimize over positive semidefinite (PSD) matrices which partially characterizes many aspects of quantum information where quantum states, channels, and measurements are all required to be representable as a PSD matrix.
As such, SDP formulations of quantum information theoretic problems have found widespread use in approximating optimal solution values to decide state separability \cite{DPShierarchy} and to the class of commutative strategies to nonlocal games \cite{NPAhierarchy}, along with providing exact algebraic characterizations for antidistinguishability of quantum states \cite{Johnston2025tightbounds}.

In both of the above SDP approximation cases relating to state separability and the set of quantum correlations, no unconditional $\omega(1)$-round integrality gaps were known prior to work in \cite{Harrow_2019}.
This project will review work by Harrow, Nataranjan, and Wu which unconditionally shows the above results independent of the Exponential Time Hypothesis, and discuss implications of their work in quantum information and complexity theory along with recent extensions.

\section{Main Results}
\label{sec:mainresults}
\subsection{Notation and Conventions}
\label{sec:notation}
Throughout, calligraphic letters such as $\mcH, \mcK$ will denote Hilbert spaces.
As is the convention in quantum information, subscripts and occasionally superscripts are used to denote membership.
For example, a pure state belonging to $\mcH_A \otimes \mcH_B$ may be written as $\ket{\psi}_{AB}$.
Additionally, a general quantum state represented by elements of $\Dens(\mcH)$ are trace 1 operators acting on $\mcH$.
In an abuse of notation we write $\rho \in \mcH$ to denote a state $\rho$ belonging to the density matrices over $\mcH$.
In general by identifying each Hilbert space with its dimension, we allow the dimension of $\mcH_A$ to be notated by $d_A$.

A state $\rho_{AB} \in \mcH_A \otimes \mcH_B$ is said to be \textit{product state} if there exist density matrices $\rho_A \in \mcH_A, \rho_B \in \mcH_B$ for which $\rho_{AB} = \rho_A \otimes \rho_B$.
The set of \textit{separable states} is then the convex hull of product states.
In particular:
\[\Sep(d_A, d_B) := \conv(\{\rho \otimes \sigma : \rho \in \mcH_A, \sigma \in \mcH_B\})\]
For an arbitrary Hermitian operator $M$ acting on $\mcH_A \otimes \mcH_B$ with $\norm{M}_{\text{op}} \le 1$, define
\[h_{\Sep(d_A, d_B)}(M) := \max_{\rho\in \Sep(d_A, d_B)} \Tr(M\rho)\]
% Since the optimal value is achieved on an extreme point (as $\Sep(d_A, d_B)$ is compact, convex in finite dimensions) this
Furthermore, we shorthand $\Sep(d, d)$ by $\Sep^2(d)$.

A \textit{quantum channel} refers to a completely positive, trace preserving map.

We begin with a list of theorems stated informally.
\subsection{Informal Statements of Main Results}
\label{sec:informalmain}
\begin{theorem}
Any SDP relaxation of $h_{\Sep^2(d)}(\cdot)$ achieving $1/\poly(d)$ accuracy must have $\ge d^{\tilde{\Omega}(\log d)}$ variables.
\end{theorem}
In fact, the paper rules out a larger class of approximations which is referred to as \textit{SDP extended formulations}.

As a corollary of the above, the paper also unconditionally affirms a version of Watrous' ``no approximate disentangler'' conjecture.
Say a quantum channel $\mcN$ is an approximate disentangler from $D$ to $d\times d$ dimensions if $\Image(\mcN) \approx_\eps \Sep(d, d)$ where $\eps \in O(1/\poly(d))$.
\begin{theorem}
If $D = \dim(\mcH), d = \dim(\mcK)$ and $\Lambda : \Dens(\mcH) \to \Dens(\mcK\otimes \mcK)$ is an approximate disentangler, then
\[D \ge d^{\tilde{\Omega}(\log d)}\]
\end{theorem}
While the above contributions relate to the DPS hierarchy for approximating the set of separable states, Harrow et al. also give lower bounds for the NPA hierarchy for calculating optimal non-local entangled game values.

Briefly, a non-local game $\mcG = (\mcQ_A, \mcQ_B, \mathcal{O}_A, \mathcal{O}_B, V, \pi)$ consists of questions $\mcQ_A, \mcQ_B$ and answers $\mathcal{O}_A, \mathcal{O}_B$ given to Alice and Bob respectively, with victory conditions given by $V: \mcQ_A \times \mcQ_B \times \mathcal{O}_A \times \mathcal{O}_B \to \{0,1\}$ and probability distribution $\pi$ over the question set $\mcQ_A \times \mcQ_B$. 
Alice and Bob are allowed to decide on a strategy beforehand, thereafter a referee then samples questions following $\pi$ and asks them to Alice and Bob separately who are then no longer allowed to communicate.
The value of a game is then the best probability Alice and Bob have to win under a set of allowable classical or quantum correlations.
The classical value is denoted by $\omega_c$, whereas if Alice and Bob are allowed to share a finite-dimensional state which they perform local measurements on, the tensor product value is denote by $\omega_q$.
Local measurements commute, from which the appropriate generalization is to commuting measurements on a possibly infinite dimensional state, with optimal value denoted by $\omega_{qc}$.

It has been known since work by Bell \cite{PhysicsPhysiqueFizika.1.195} that strategies in which Alice and Bob share an entangled state which they then perform local measurements on can outperform classical strategies.
Further work by Tsirelson \cite{tsirelson1993some} bounds exactly how much better quantum strategies can outperform classical correlations (by a multiplicative factor of the Grothendieck constant relating to the dimension of space in which the quantum strategy lives) but also posed conjectures on whether the set of tensor product correlations was equal to the set of commuting correlations.
Tsirelson's strong conjecture was disproved by Slofstra \cite{Slofstra_2019}, while the weak case known to be equivalent to the Conne's embedding problem \cite{MR2790067} stood until a complexity theoretic approach (in a move which frustrated many operator algebraists) refuted it \cite{ji2021mip}. To my best knowledge (a couple of Google searches) an explicit construction of a non-embeddable tracial von Neumann algebra has yet to be found.

In relation to the above discussion, Harrow et al. also give unconditional lower bounds for the error of the ncSoS hierarchy along with minimal SDP dimensions achieving a desired accuracy:

\begin{theorem}
  There exists a sequence of non-local games $\mcG_n$ for which the quantum commuting value of the games $\omega_{qc}(\mcG_n) \le 1 - c/n^2$ for some constant $c$ but the ncSoS hierarchy believes $\omega_{qc}(\mcG_n) = 1$ up to level $m = \Omega(n)$.
\end{theorem}

\begin{theorem}
  There exists a sequence of non-local games $\mcG_n$ for which any SDP relaxation approximating $\omega_{qc}(\mcG_n)$ with error $O(1/n^2)$ has dimension $\ge n^{\log n/\poly \log \log n}$.
\end{theorem}

In order to achieve the above results, Harrow et al. set up a framework to ``reduce'' integrality gap results from one problem to another.
The reduction framework introduced provides a method allowing problems with known integrality gap bounds to be reduced to intermediate problems in smaller steps which may be more amenable to analysis which preserves the low degree solutions to produce tighter bounds.
A more detailed analysis of these ideas will be given in section \ref{sec:techniques}.

Using this above reduction framework, Harrow, Nataranjan, and Wu also extend results by Lee, Raghavendra, and Steurer in \cite{lee2014lowerboundssizesemidefinite} from boolean domains to non-boolean domains such as the hypersphere.
Additionally, previous unconditional hardness of approximation results \cite{tulsiani2009csp,odonnell2014hardnessrobustgraphisomorphism} matching bounds given by either the Exponential Time Hypothesis or the Unique Games Conjecture operate solely over boolean domains.
The paper produces the first unconditional results for SoS hardness problems over non-boolean domains.

My belief is that this last contribution giving a framework to reduce known unconditional integrality gaps for well studied problems to unconditional integrality gaps other problems is the strongest contribution of the paper, and the previous 4 theorems only demonstrate an application of the approach.
A more detailed criticism will be given later in section \ref{sec:criticisms}

\section{Techniques and Proof Ideas}
\label{sec:techniques}

\subsection{The Category of Optimization Problems}
\label{sec:optimizationdef}
We give a rigorous definition to the class of optimization problems considered. 
It should be noted that the original definition provided by Harrow et al. is not of a categorical flavour, but I believe a categorical lens offers a cleaner presentation of the main ideas encapsulating the definition while also providing additional tools and conjectures.

\subsubsection{Objects:}
The objects of the category of optimization problems will of course be a suitable definition of optimization problems.
This leads to a definition which suitably encapsulates the ``size'' of optimization problems as a family.
\begin{definition}\label{def:optprob:obj}
  An object $A$ in the category of optimization problems is a sequence parameterized by $n \in \En$ of tuples $(\mcP_n^A, \Delta_n^A)$ of which:
  \begin{itemize}
    \item \textbf{Feasible Set}: $\mcP_n^A$ denotes a set of feasible solutions at each size $n$
    \item \textbf{Instances}: $\Delta_n^A$ represents a set of optimization problem instances (\ie objective functions) as maps $\Phi: \mcP_n^A \to [0,1]$
  \end{itemize}
  Associated with every $n, \Phi \in \Delta_n^A$, it is also natural to consider the optimal value of the instance denoted by
  \[\OPT(\Phi) \coloneq \max_{x\in \mcP_n^A} \Phi(x)\]
\end{definition}
This definition encapsulates a very broad definition of optimization problems, and in particular admits an (uncountably) infinite number of characterizations by a reindexing of the sequence which defines such a problem.
To this extent, we consider only ``well-formed'' objects where the sequence is indexed correctly by a size which admits a run-time theoretic interpretation.
For example, in the context of semi-definite programs we may consider each $\mcP_n^{\text{SDP}}$ to be the set of all $n\times n$ PSD matrices.
Note that in general any affine function on the set of $n\times n$ PSD matrices does not have range contained in $[0,1]$, but we can simply consider $[0,1]$ isomorphic to the 2-point compactification of the reals and the instances in the above definition will then just be the pullback of any affine function by this isomorphism ($\arctan$, for example).

Combined with the above observation that the instances having co-domain $[0,1]$ is equivalent to the instances having co-domain of the extended reals, we draw notice to that the above definition is robust.
This is because we may consider any class of optimization instances at size $n$(which may not all have the same feasible region) and define the feasible region at size $n$ to be the union over all such instance feasible regions, and encode infeasibility as having objective value $0$.

The restriction to having co-domain $[0,1]$ is natural, however, when considering the specialization to the class of problems at hand.

\begin{definition}
  A \textit{polynomial} optimization problem is one where each $\mcP_n^A$ is a variety (affine, in the algebraic geometry sense) of $\Ree^n$ given by at $m = m(n)$ polynomial equality constraints.
\end{definition}
A special case of a polynomial optimization problem is:
\begin{definition}
  A \textit{boolean} optimization problem is given by the polynomial constraints $x_i^2 = 1$. As such $\mcP_n^{bool}$ is the hypercube $\{-1,1\}^n$ for every $n$.
\end{definition}
\begin{definition}
  A \textit{constraint satisfaction problem} is a boolean optimization problem where each instance is defined by a collection of \textit{clauses} $\{h_1, ..., h_k\}$ where each $h_i : \{\pm 1\}^n \to \{0,1\}$ is boolean, with objective value counting the normalized number of clauses satisfied for some assignment of variables.
\end{definition}

\section{Criticisms}
\label{sec:critisms}
uwu
% \section{Conclusion and Outlook}
% \label{sec:outlook}
% \lipsum[1-2]

% \subsubsection*{Acknowledgments}

% Identification of funding sources and other support, and thanks to individuals and groups that assisted in the research and the preparation of the work should be included in an acknowledgment section, which is placed just before the reference section in your document

%%% ===============================================================================
%%% Bibliography
%%% ===============================================================================

% In the bibliography, use \texttt{\textbackslash textsuperscript} for \enquote{st}, \enquote{nd}, \ldots:
% E.g., \enquote{The 2\textsuperscript{nd} conference on examples}.
% When you use \href{https://www.jabref.org}{JabRef}, you can use the clean up command to achieve that.
% See \url{https://help.jabref.org/en/CleanupEntries} for an overview of the cleanup functionality.

\renewcommand{\bibsection}{\section*{References}} % requried for natbib to have "References" printed and as section*, not chapter*
% Use natbib compatbile splncs04nat style.
% It does provide all features of splncs04.bst, but is developed in a clean way.
% Source: https://github.com/tpavlic/splncs04nat
\bibliographystyle{splncs04nat}
\begingroup
  \microtypecontext{expansion=sloppy}
  \small % ensure correct font size for the bibliography
  \bibliography{paper}
\endgroup

% Enfore empty line after bibliography
% \ \\
% %
% \noindent
% All links were last followed on October 5, 2020.

%%% ===============================================================================
%\appendix
%\addcontentsline{toc}{chapter}{APPENDICES}

%\listoffigures
%\listoftables
%%% ===============================================================================

%%% ===============================================================================
%\section{My first appendix}\label{sec:appendix1}
%%% ===============================================================================
\end{document}
